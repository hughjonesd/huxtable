\documentclass{article}
\usepackage{array}
\usepackage{caption}
\usepackage{graphicx}
\usepackage{siunitx}
\usepackage[normalem]{ulem}
\usepackage{colortbl}
\usepackage{multirow}
\usepackage{hhline}
\usepackage{calc}
\usepackage{tabularx}
\usepackage{threeparttable}
\usepackage{wrapfig}
\usepackage{adjustbox}
\usepackage{hyperref}
% These are LaTeX packages. You can install them using your LaTex management software,
% or by running `huxtable::install_latex_dependencies()` from within R.
% Other packages may be required if you use non-standard tabulars (e.g. tabulary).

\pagenumbering{gobble}

\begin{document}


  \providecommand{\huxb}[2]{\arrayrulecolor[RGB]{#1}\global\arrayrulewidth=#2pt}
  \providecommand{\huxvb}[2]{\color[RGB]{#1}\vrule width #2pt}
  \providecommand{\huxtpad}[1]{\rule{0pt}{#1}}
  \providecommand{\huxbpad}[1]{\rule[-#1]{0pt}{#1}}

\begin{table}[ht]
\begin{raggedright}
\begin{threeparttable}
\captionsetup{justification=raggedright,singlelinecheck=off}
\caption{Table 1: position="left"}
 \setlength{\tabcolsep}{0pt}
\begin{tabular}{l}


\hhline{>{\huxb{0, 0, 0}{1}}-}
\arrayrulecolor{black}

\multicolumn{1}{!{\huxvb{0, 0, 0}{1}}l!{\huxvb{0, 0, 0}{1}}}{\huxtpad{6pt + 1em}\raggedright \hspace{6pt} position=left \hspace{6pt}\huxbpad{6pt}} \tabularnewline[-0.5pt]


\hhline{>{\huxb{0, 0, 0}{1}}-}
\arrayrulecolor{black}
\end{tabular}
\end{threeparttable}\par\end{raggedright}

\end{table}





  \providecommand{\huxb}[2]{\arrayrulecolor[RGB]{#1}\global\arrayrulewidth=#2pt}
  \providecommand{\huxvb}[2]{\color[RGB]{#1}\vrule width #2pt}
  \providecommand{\huxtpad}[1]{\rule{0pt}{#1}}
  \providecommand{\huxbpad}[1]{\rule[-#1]{0pt}{#1}}

\begin{table}[ht]
\begin{centerbox}
\begin{threeparttable}
\captionsetup{justification=centering,singlelinecheck=off}
\caption{Table 2: position="center"}
 \setlength{\tabcolsep}{0pt}
\begin{tabular}{l}


\hhline{>{\huxb{0, 0, 0}{1}}-}
\arrayrulecolor{black}

\multicolumn{1}{!{\huxvb{0, 0, 0}{1}}l!{\huxvb{0, 0, 0}{1}}}{\huxtpad{6pt + 1em}\raggedright \hspace{6pt} position=center \hspace{6pt}\huxbpad{6pt}} \tabularnewline[-0.5pt]


\hhline{>{\huxb{0, 0, 0}{1}}-}
\arrayrulecolor{black}
\end{tabular}
\end{threeparttable}\par\end{centerbox}

\end{table}





  \providecommand{\huxb}[2]{\arrayrulecolor[RGB]{#1}\global\arrayrulewidth=#2pt}
  \providecommand{\huxvb}[2]{\color[RGB]{#1}\vrule width #2pt}
  \providecommand{\huxtpad}[1]{\rule{0pt}{#1}}
  \providecommand{\huxbpad}[1]{\rule[-#1]{0pt}{#1}}

\begin{table}[ht]
\begin{raggedleft}
\begin{threeparttable}
\captionsetup{justification=raggedleft,singlelinecheck=off}
\caption{Table 3: position="right"}
 \setlength{\tabcolsep}{0pt}
\begin{tabular}{l}


\hhline{>{\huxb{0, 0, 0}{1}}-}
\arrayrulecolor{black}

\multicolumn{1}{!{\huxvb{0, 0, 0}{1}}l!{\huxvb{0, 0, 0}{1}}}{\huxtpad{6pt + 1em}\raggedright \hspace{6pt} position=right \hspace{6pt}\huxbpad{6pt}} \tabularnewline[-0.5pt]


\hhline{>{\huxb{0, 0, 0}{1}}-}
\arrayrulecolor{black}
\end{tabular}
\end{threeparttable}\par\end{raggedleft}

\end{table}



\end{document}